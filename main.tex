\documentclass[]{cv-style} % Add 'print' in [] to get print-view
\usepackage{graphicx}
\begin{document}
\header{Pablo F.}{ Ordoñez-Ordoñez}
\begin{aside}
\section{DOCENTE INVESTIGADOR {CIS-UNL}} %Rol en este proyecto
\includegraphics[width=4cm]{foto}
\section{Contacto}
Documento Nacional: 1103674733
%Calle Sucre y Celica
Loja, Ecuador
~
+593 992693531
~
pfordonez@unl.edu.ec
~
\section{Títulos}
TERCER NIVEL:
Ingeniero en Sistemas Informáticos y Computación
CUARTO NIVEL:
Máster universitario en Ingeniería del Software para la Web
Máster en Sistemas de Información SAP ERP
OTRA FORMACIÓN:
Doctorando en Ciencias y Tecnologías de la Computación para Smart Cities
Maestrante en Ingeniería Computacional y Sistemas Inteligentes 
~
\section{Redes Sociales}
\small
\url{https://www.researchgate.net/profile/Pablo_Ordonez-Ordonez}
~
\url{https://orcid.org/0000-0001-8079-7694}
~
\url{https://www.linkedin.com/in/pfordonez/}
~
\end{aside}
\small
\section{Perfil Profesional}
  \vspace{-0.3cm}
\small
Como Ingeniero en Sistemas Informáticos y Computación, mi formación académica, humana y laboral se ha enfocado al desarrollo del conocimiento en algunas universidades del país y la implementación de propuestas útiles en el campo de la profesión. Tengo experiencia con la ingeniería del software, sistemas de información, ingeniería computacional y sistemas inteligentes. Soy un profesional comprometido con la investigación en las ciencias computacionales y sus aplicaciones en los ambientes inteligentes, el liderazgo y el trabajo en equipo.

\section{Experiencia Profesional}
\vspace{-0.3cm}
\begin{entrylist}
\small
\entry
  {2012 - actual}
  {Universidad Nacional de Loja, Carrera Computación, FEIRNNR}
  {Loja, Ecuador}
  {\jobtitle{Docente Investigador} }
\entry
  {2015}
  {Universidad Técnica del Norte, Maestría en Ingeniería del Software}
  {Ibarra, Ecuador}
  {\jobtitle{Docente Postgrado} }

\entry
  {2014}
  {Yachay EP}
  {Urcuqui, Ibarra. Ecuador}
  {\jobtitle{Analista de Sistemas} }
\entry
  {2011}
  {SCL Consulting}
  {Madrid, España}
  {\jobtitle{Consultor SAP ERP - Abap} }
 \entry
  {2010}
  {Universidad Técnica Particular de Loja}
  {Loja, Ecuador}
  {\jobtitle{Desarrollador de Software} }
 \entry
  {2009}
  {CDI Chile}
  {Santiago, Chile}
  {\jobtitle{Consultor TICs} }
\entry
  {2006}
  {Constructora Velez\&Vanegas }
  {Loja, Ecuador}
  {\jobtitle{Consultor TICs \& Desarrollador de Software} }
\end{entrylist}

\section{Publicaciones} %En formato cita APA, las últimas o las mas relevantes
\vspace{-0.3cm}
\begin{entrylist}
\entry
    {2013 - actual}
    {10 publicaciones: {\normalfont 2} en revistas, {\normalfont 8} en libros de congresos}
    {Scopus - Latindex}
    {\jobtitle{Resumen:}

\begin{itemize}
\item \small Ordonez-Ordonez, P. F., Herrera-Loaiza, D. D., \& Figueroa-Diaz, R. (2018, December). Vulnerabilities in Banking Transactions with Mobile Devices Android: A Systematic Literature Review. In Technology Trends: 4th International Conference, CITT 2018, Babahoyo, Ecuador, August 29–31, 2018, Revised Selected Papers (Vol. 895, p. 104). Springer.
\item Ordonez-Ordonez, P. F., Quizhpe, M., Cumbicus-Pineda, O. M., Salazar, V. H., \& Figueroa-Diaz, R. (2018, August). Application of Genetic Algorithms in Software Engineering: A Systematic Literature Review. In International Conference on Technology Trends (pp. 659-670). Springer, Cham.
\item Pineda, A. I. G., \& Ordonez, P. F. O. (2014). Identificacion de Factores en la Desercion y Reprobacion Universitaria. In Actas del VI Congreso Internacional sobre Aplicación de Tecnologias de la Informacion y Comunicaciones Avanzadas (ATICA 2014): Universidad de Alcala de Henares (Espana), 29-31 de octubre de 2014 (pp. 473-480).
\item Cumbicus-Pineda, O. M., Ordonez-Ordonez, P. F., Neyra-Romero, L. A., \& Figueroa-Diaz, R. (2018, August). Automatic Categorization of Tweets on the Political Electoral Theme Using Supervised Classification Algorithms. In International Conference on Technology Trends (pp. 671-682). Springer, Cham.
\item Suntaxi-Sarango, M. C., Ordonez-Ordonez, P. F., \& Pesantez-Gonzalez, M. A. (2018). Applications of Deep Learning in Financial Intermediation: A Systematic Literature Review. KnE Engineering, 3(9), 47-60.
\end{itemize}
}
\end{entrylist} 
\section{Proyectos de Investigación}
\vspace{-0.3cm}
\begin{itemize}
    \item  \small Ambiente inteligente para el macro laboratorio de formación conjunta en la Facultad de Energía de la Universidad Nacional de Loja: SmartLab, monto de \$48000 financiado por la UNL.
    \item \small Diagnóstico y prospectiva de las energías que interactúan en la zona 7 del Ecuador, monto de \$30000 financiado por la UNL.
\end{itemize}
\end{document}